\chapter{Build and Deployment Pipeline}

\section*{Context: Security in CI/CD}

\textbf{Build and deployment pipeline}

Per automatizzare tutte le operazioni di testing, deployment e build è stata predisposta una github action che viene eseguita ad ogni push.

\section*{Overview}

La nostra pipeline ‘maven.yml’ garantisce che ogni modifica sia pronta e sicura per la produzione.

\begin{itemize}
    \item \textbf{trigger}: Il workflow si attiva ad ogni push sul branch principale (main)
    \item \textbf{ambiente}: L’intero workflow gira su una macchina virtuale basata su ‘ubuntu-latest’
\end{itemize}

\textbf{flusso:}

Nella nostra pipeline vengono eseguiti più step, in ordine:
\begin{itemize}
    \item Checkout del codice e setup Java 19
    \item Inizializzazione del database con import dei dati di test
    \item Build, test e mutation coverage’
    \item Upload dei risultati su codecov
    \item Setup delle credenziali per dockerhub
    \item Upload dell’immagine su dockerhub
\end{itemize}

\section*{Steps nel dettaglio}

In questo paragrafo analizziamo alcuni passi fondamentali della nostra github action.

\textbf{Mysql:}

all’interno della github action viene istanziato un database di test con dati di test. Questo è necessario perchè altrimenti la maggior parte dei test fallirebbero per mancata connessione al db.  
Questa soluzione permette di eseguire i test su un ambiente totalmente isolato, riproducibile e modificabile.

\vspace{0.3cm}

\textbf{Build, Unit Testing e Mutation Testing:}

Tramite il comando ‘mvn clean test -P pitest’ riusciamo a buildare, lanciare gli unit test e mutation test sulla nostra applicazione.  
Tramite il profilo maven ‘pitest’ vengono introdotti dei mutanti nei package model.beans, model.dao e services.

I risultati dei test vengono successivamente caricati sulla piattaforma ‘Codecov’ che ci permetterà di visualizzare i risultati e gestirà il fallimento o successo della push grazie alla key ‘\verb|fail_ci_if_error|: true’

\vspace{0.3cm}

\textbf{Sicurezza:}

All’interno della nostra github action non troveremo credenziali ma richiami a secrets salvate nella repo github, come ad esempio

\begin{verbatim}
- name: Log in to Docker Hub
     uses: docker/login-action@v3
     with:
       username: ${{ secrets.DOCKER_USERNAME }}
       password: ${{ secrets.DOCKER_PASSWORD }}
\end{verbatim}
