\chapter{Mutation Testing}

\section*{Context: PiTest}

\textbf{Mutation testing}

\vspace{0.3cm}

All’interno del nostro progetto utilizzeremo il mutation testing grazie a PiTest che ci permette di superare i limiti della semplice coverage per la quale utilizziamo jacoco ed individuare bug nel codice sorgente.

\vspace{0.4cm}

\textbf{Tool utilizzato: pitest}

\vspace{0.2cm}

Come tool per il mutation testing è stato selezionato pitest come visto a lezione.  
Permette di essere integrato facilmente con Maven e supporta Junit 5.  
La versione specifica utilizzata nel nostro progetto e funzionante è la 1.15.0.

\section*{Configurazione tecnica ed utilizzo}

All’interno del nostro pom.xml abbiamo inserito un profilo Maven chiamato pitest che permette di eseguire il mutation testing lanciando il comando:

\begin{center}
\texttt{mvn test-compile pitest:mutationCoverage -P pitest}
\end{center}

Le classi testate sono i beans i dao e i services.

\vspace{0.3cm}

Per l’analisi abbiamo utilizzato i mutatori standard di pitest:
\begin{itemize}
    \item Inversione dellec ondizioni
    \item Sostituizione dei valori di ritorno
    \item Modifica degli operatori aritmetici
\end{itemize}

I risultati li troveremo nella cartella \texttt{target/pit-reports}.

\section*{Primi risultati}

I primi risultati evidenziano scarsi risultati per quanto riguarda i test su dao e services.

\vspace{0.4cm}

\begin{center}
\resizebox{\textwidth}{!}{
\begin{tabular}{|l|c|c|c|c|}
\hline
\textbf{Nome Pacchetto} & \textbf{Numero di Classi} & \textbf{Line Coverage} & \textbf{Mutation Coverage} & \textbf{Test Strength} \\
\hline
model.beans & 9 & 99\% (292/294) & 97\% (136/140) & 97\% (136/140) \\
model.dao & 12 & 55\% (433/788) & 30\% (103/344) & 35\% (103/294) \\
services & 19 & 60\% (339/565) & 45\% (96/214) & 47\% (96/203) \\
\hline
\textbf{TOTALE PROGETTO} & 51 & 53\% (1064/1999) & 42\% (335/797) & 52\% (335/637) \\
\hline
\end{tabular}
}
\end{center}

\vspace{0.3cm}

Il problema è stato risolto aumentando i test e rendendoli più specifici.

\section*{Risultati finali}

\vspace{0.4cm}

\begin{center}
\resizebox{\textwidth}{!}{
\begin{tabular}{|l|c|c|c|c|}
\hline
\textbf{Nome Pacchetto} & \textbf{Numero di Classi} & \textbf{Line Coverage} & \textbf{Mutation Coverage} & \textbf{Test Strength} \\
\hline
model.beans & 9 & 99\% (292/294) & 97\% (136/140) & 97\% (136/140) \\
model.dao & 12 & 77\% (606/788) & 50\% (173/344) & 59\% (173/294) \\
services & 19 & 81\% (456/565) & 83\% (177/214) & 87\% (177/203) \\
\hline
\textbf{TOTALE PROGETTO} & 51 & 68\% (1354/1999) & 61\% (486/797) & 76\% (486/637) \\
\hline
\end{tabular}
}
\end{center}